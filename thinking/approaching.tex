\chapter{Attitudes}

The way we think depends on our attitude. Let's start by looking at the attitudes of someone who is good at thinking and someone who is bad at thinking. As you read this section, \textbf{consider} how you want to work on this program.

\section*{Bad Attitudes}

If you have not been taught how to think, you might lack confidence. This can lead you to think things like:

``Thinking is not important'' \\
``Thinking is boring and confusing and never gets me anywhere'' \\
``Solving problems is too difficult'' \\
``Thinking is only for geniuses. Other people act without thinking''

These attitudes keep you from reaching your potential and reasoning about problems. But there is another type of bad attitudes. You could be arrogant about your thinking and misunderstand the purpose of thinking. This can lead you to think things like:

``Thinking is easy\ldots Just look at the problem and make up your mind'' \\
``I am always right, so I have no trouble defending my ideas'' \\
``The purpose of thinking is proving other people wrong'' \\
``If you are good at thinking, you will never be wrong'' \\
``There is one right answer to everything'' \\
``People who don't understand are wrong or stupid''

While you probably wouldn't say any of these things out loud, you may have attitudes that keep you from thinking properly.


\section*{Good Attitudes}

\textbf{Everyone can think} \\
Thinking is for everyone, not just smart or clever people. Everyone can practice thinking and become an excellent thinker.

\textbf{Everyone has to think} \\
Thinking is an important part of life. No one can make it through life without thinking, and thinking well will help you through life.

\textbf{Thinking is a skill that can be developed} \\
Thinking isn't like the color of your eyes. You can improve your thinking skills.

\textbf{I am a thinker} \\
It doesn't matter if you're very good or very bad at thinking. Everyone is a thinker, and seeing yourself as a thinker will help you think better.

\textbf{I can get better at thinking over time} \\
Even the world's best thinkers can get better and better.

\textbf{Thinking might take work} \\
It isn't enough to suppose all your thinking will be good. Sometimes you will even need to use a tool or a structure to think about a problem. Thinking is not always automatic.

\textbf{It is possible to simplify complicated things} \\
Great thinkers know that they can break a problem down into small pieces.

\textbf{Take one small step at a time} \\
There is usually no harm in taking a small step when solving a problem.

\textbf{Separate your ego from your thinking} \\
You are not your thoughts. Look at your thinking objectively.

\textbf{The purpose of thinking is not always to be right} \\
The purpose of thinking is to have ideas and get better at thinking. If you always have to be right, you will not make progress. Great thinkers are often wrong.

\textbf{Learning and listening are key parts of thinking} \\
Thinking is more about observing than talking.

\textbf{Humility marks an excellent thinker} \\
Great thinkers realize that they may be wrong, and they learn from others.

\textbf{Thinking should be constructive, not negative} \\
Attacking a person or idea does not prove the person or idea is wrong. Basing your thinking on constructive facts will allow you to reach a correct conclusion without burning bridges.

\textbf{Explore a subject; don't argue about it} \\
If a purpose of a conversation is to deepen understanding about a subject, both parties can explore a subject more productively than if they argue about it.

\textbf{The other party has something useful to say} \\
Instead of looking for points to attack, see what is of value in a differing opinion.

\textbf{People are often right according to their own perception} \\
How you see the world and the experiences you've had will always shape how you think. Try to see what may cause someone to think a certain way, and try to identify those influences in yourself.

\textbf{Creativity is possible} \\
Creativity is not some special gift that only a select few have. You can put in effort and come up with new ideas.

\textbf{Try it!} \\
Great thinkers try out tentative ideas without worrying about being wrong.

\textbf{There may be alternatives you haven't thought of} \\
Never think you covered all the possible ways to tackle a problem.

\textbf{Avoid dogmatism} \\
If your idea is good enough, you don't need the dogmatism. If it is not, the dogmatism is misplaced. Try saying instead, ``Based on my understanding, it seems that\ldots''

\newpage

\section*{Attitude Exercises}

\begin{enumerate}
	\item What does the word \textit{attitude} mean (in your own words)? \\\\
	\item What is your attitude toward sports? Music? Reading? Orange juice? \\\\
	\item Read the list of bad attitudes. Why might some people have bad attitudes? \\\\
	\item What other bad attitudes would you add to the list? Why? \\\\
	\item Pick 4 good attitudes and write why they are good attitudes.
		\begin{itemize}
			\item 
			\item
			\item 
			\item 
		\end{itemize}
	\item Mark the 3 good attitudes you think are most useful for you.
	\item How would you combine the good attitudes into a shorter list? \\\\
	\item \textbf{Discussion:} What would you add to the list of good attitudes?
	\item \textbf{Discussion:} Why is a good attitude toward thinking important?
	\item \textbf{Discussion:} How will improving your thinking help you in your life?
\end{enumerate}



\chapter{Thinking Hats}

\ibox{\textbf{Before You Begin}\\
	Balance a book on your head. Then unwrap a piece of candy with your right hand and turn the page with your left hand.}

Doing a lot of things at the same time is almost impossible. When we think, we often try to do too much at once. We look at the facts, try to build a logical argument, take our emotions into account, put in a new idea, and see if the idea will work all at the same time. This can cause us to become confused and only do one of these things well.

Have you heard the expression ``put on your thinking cap''? To help you do one type of thinking at a time, we will use six thinking hats. Instead of doing everything all at once, we will only wear one hat at a time. This is because some chemicals in your brain act differently when you are being creative, positive, and negative. Putting on a different hat for different thinking tasks can let you use your whole brain to solve a problem, rather than just a part of it.

Printing something in color usually uses very small dots of four different colors (cyan, magenta, yellow, and black) to create a page with all the colors. In the same way, you can use the six hats to solve all kinds of problems, even though you only tackle a small part of the problem at once.

\newpage

\section*{Making Your Thinking Hats}

\ibox{\textbf{Activity}\\ Get six pieces of paper and fold them into hats. Label your hats as described below. Make each hat a different color to help distinguish them.}

\textbf{Fact Hat} \\
This hat is for facts, figures, and information. When you wear this hat, ask what information you have and what information you still need.

\textbf{Feeling Hat} \\
This hat is for emotions, feelings, hunches, and intuition. When you wear this hat, ask how you feel about the matter.

\textbf{Caution Hat} \\
This hat is for caution, judgment, and fitting the facts. When you wear this hat, ask if it will work, if it is safe, and if it is possible.

\textbf{Optimism Hat} \\
This hat is for advantages, benefits, and savings. When you wear this hat, ask what the benefits are and why the idea is good.

\textbf{Idea Hat} \\
This hat is for new ideas, proposals, suggestions, and exploration. When you wear this hat, ask what new and different solutions you can come up with.

\textbf{Meta Hat} \\
This hat is for thinking about thinking. When you wear this hat, take a step back and look at your process of thinking.

\newpage

\section*{About the Hats}

Wearing the hats (or thinking about them) allows you to separate yourself from your thinking process. You can choose to put on your Idea Hat and come up with new ideas for 5 minutes. While you have the Idea Hat on, you don't use any of your other hats, so you can come up with ideas without judging them. When you're done, putting on your caution hat can help you decide which ideas are possible.

Role-playing with hats lets you put on a different performance for each type of thinking you do. Rather than feeling pressured to come up with an idea on-the-spot, you can divide the process into different types of thinking.

\section*{Hats as a Tool}

You can choose to put on hats in any order that makes sense to you. Use the Meta Hat to organize your thinking process how you want to.

It can be helpful to use different notebooks or documents to work with each hat. For example, write your feelings in a different place than your new ideas.

These six hats are an \textit{attention-directing tool}. Putting on a specific hat can help you direct your attention to a specific thought or part of your solution.

\newpage 

\section*{Thinking Hat Exercises}

\begin{enumerate}
	\item In what kind of situations would the thinking hats be useful? ~\\~\\~\\
	\item Do you think the thinking hats will be useful to you? Why or why not? ~\\~\\~\\
	\item Why might some people not want to use thinking hats? ~\\~\\~\\
	\item If you had to suggest more hats, what types of thinking would these hats cover? ~\\~\\~\\
	\item For each comment, which hat was the speaker likely wearing? \\
		``My car goes 26 miles on one gallon of gasoline'' \\
		``I'm tired, let's just throw the project away and start over.'' \\
		``Raising the price of alcohol could cause people to drink less.'' \\
		``We should think about this from a different perspective.'' \\
		``Taking the job could save me hundreds of dollars!''
\end{enumerate}

\chapter{Using Thinking Hats}

\ibox{\textbf{Before You Begin} \\
	Collect the hats you made for the previous chapter. Wear the appropriate hat as you study each section.}

In this chapter, we're going to take a deeper look at each of the hats. This is a long section, so it may be useful to break it up into smaller pieces to study.

\section*{Fact Hat}

The fact hat focuses on neutral information. While you use white hat thinking, focus only on logical facts rather than ideas, opinions, or feelings. When you use the white hat, ask yourself three questions:

1. What information do I have? \\
2. What information am I missing? \\
3. How can I get the information I still need?

\subsection*{The Information You Have}
First, lay out all of the information you have. This information can include facts, figures, lists, statistics, definitions, or other details. It may also be your own personal knowledge or experience. It is important to label this information appropriately. For example, if it is your experience, label it ``in my experience'' or ``as far as I know''. It is wise to keep track of where you get your information from by writing the source where you write the information.

You can also read between the lines to get more information. Every good detective picks up clues that other people haven't noticed.

Just like when you sometimes buy a bad avocado in the grocery store, sometimes you run into bad information. While wearing your fact hat, state clearly what type of information you have. For example, your notebook may look like this:


\notebook{
\textbf{Scientific Fact:} A crocodile cannot stick its tongue out. (National Institute of Environmental Health Sciences).

\textbf{In my experience:} I've never seen a crocodile's tongue.

\textbf{My guess:} A crocodile's tongue isn't long enough to stick out.
}

Keeping track of where you get information can be very helpful. It can help you avoid using your personal ideas as scientific evidence, and it can be useful to people who hear your ideas if they want to do similar research.

\subsection*{Missing Information}
After you examine the information you have, find out what information is missing. Try to find any gaps in your information. Do you have enough information to make a decision? If you don't have enough information yet, what other information do you need?

Define the information you need as clearly as possible. While it is always nice to have more information, try to find out what you really need. You may need more information to choose between two possible explanations, to select the best course of action, or to know if something will meet your needs.

\subsection*{Getting Information}
Listening carefully is part of Fact Hat thinking. You can get information in a number of ways, including reading a book, asking an expert, or searching reliable sources on the internet. 

The most useful way to get information is to ask questions. Knowing the right questions to ask is an important part of thinking. There are many kinds of questions. If you need to check if something is accurate, try asking a yes-or-no question. For example, ``Is it true that squirrels can jump 20 feet?'' If you need to gather specific details, try asking a detailed question. For example, ``What is the first note of Moonlight Sonata?'' If you need general information, try asking a broad question. For example, ``Can you tell me about World War II?''

\subsection*{Information and Feelings}
Sometimes, the Fact Hat and the Feeling Hat get quite close. If you are trying to look to the future, you will have to use some feelings to guess or extrapolate. If this is the case, it is important to disclose that you are using your feelings. For example, ``I feel like this product will sell''.

Discussing the feelings of others can be Fact Hat thinking, though. If you note that Mr. Smith does not like apples, you are stating a fact. However, if you argue that you do not like apples, you are using your own feelings, and that is better for Feeling Hat thinking.

\ibox{\textbf{Challenging Situation} \\
If two facts contradict each other, what should you do?

The simple answer is to take note of both facts. You may write ``Sam says a walrus is a mammal, but Sandra says it is a fish. Investigate further.''
}

\subsection*{Fact Hat Exercises}
\begin{enumerate}
	\item Can a computer do Fact Hat thinking? Why or why not? \\~\\
	\item Someone wants to do a project on your street. Do some Fact Hat thinking about the street you live on.
\end{enumerate}

\section*{Feeling Hat}

The Feeling Hat is for emotions, feelings, hunches, and intuition. In a way, it is the opposite of the Fact Hat. The Feeling Hat does not care about logical facts, but it cares about how you and others feel. 

Despite what some people say, feelings are a crucial part of thinking. Feelings come into thinking all the time. We may try to be objective, but we rarely are. Noticing how you feel about something can help you give your feelings a place in the thinking process. Feelings are valuable when we label them as feelings, but when we pretend they are something else, we can start to have problems. The Feeling Hat helps you give your feelings a clear label.

Intuition is often based on our experience. We can have a hunch or an intuition that something is the right thing to do, but we often cannot explain exactly how we came to the conclusion. Intuitions can be very valuable, but sometimes, they can be harmful. For instance, an intuition may cause someone to gamble despite the high probability of losing money.

\subsection*{The Need for Justification}
Normally, when we have a hunch or intuition, we try to give it a reasonable basis. This basis can often be false even though an intuition is valid.

The Feeling Hat lets you put forward an intuition without needing to justify it. When wearing the feeling hat, you can say, ``I have a hunch Dave is going to be good at tennis, don't ask me why.''
In fact, when wearing the Feeling Hat, you should never attempt to support or justify your intuition. Just make a note of the feeling and move on.

\subsection*{At This Moment}
The Feeling Hat is about how you feel ``at this moment''. At the start of a day, your feelings may be very different from at the end of the day.

A feeling is only valid if it is sincere. That is, you should only make note of feelings you truly have, and you should make a note of when you feel that way.
It is normal for feelings to change over time. It is acceptable to note when your feelings have changed, as that is an important part of the Feeling Hat.

It is also normal to have mixed feelings, as long as you note them as such. For instance, you might write ``I don't like the color of the bike, but I like the feel of the handlebars.''

\subsection*{Feeling Hat Exercises}
\begin{enumerate}
	\item Two boys kick a ball into someone's yard and start to fight. What Feeling Hat remarks might each boy make? \\~\\
	\item You want to pick a new hobby. Do some Feeling Hat thinking about gardening, origami, and stamp collecting. \\~\\~\\~\\~\\
	\item Put on your Feeling Hat and write 5 things you like and 5 things you don't.
\end{enumerate}

\newpage

\section*{Caution Hat and Optimism Hat}

Both the Caution Hat and Optimism Hat help you express judgment. When you wear the Caution Hat, you are concerned with truth and fit. When you wear the Optimism Hat, you are concerned with benefits. \textit{When you wear either hat, you must be completely logical.} With either hat, you need to have strong reasons for what you say. If you have no reasons, you should label your thinking as Feeling Hat thinking.

\subsection*{Caution Hat Questions}
Imagine a stern judge. The Caution Hat is likely the hat you will use the most. It will prevent you from making mistakes in thinking. It is concerned with truth and reality. The Caution Hat is the hat of critical thinking. When you wear the Caution Hat, ask the following questions.

\textbf{Is it true?} \\
When wearing the Caution hat, analyze the truth of a statement or claim. Do some extra research if you have to. Ask if the information fits all of the facts you have gathered. This is when you resolve discrepancies in information. Ask also if your conclusion follows from your evidence. Have you made a mistake? Have you justified your claim?

\textbf{Does it fit?} \\
Does this suggestion fit with your experience? Does the claim fit in the system you're working with? Does it fit the law, rules, and social customs it needs to? Does it fit your objectives, short-term and long? Does it fit with your values, ethics, and morals? Is it fair and just? If it does or does not fit, explain why.

\textbf{Will it work?} \\
Ask if the idea will work. If you decide it will not work, give reasons. Find any weakness you can with the idea.

\textbf{What are the dangers and problems?} \\
If you go ahead with the idea, what are the dangers? What problems might arise? What harmful effects could your idea have? Again, give reasons.

\subsection*{Overuse of the Caution Hat}
The Caution Hat is sometimes overused. Some people want to be cautious and negative all the time. But the Caution Hat is like salt on food. You need some salt in food, or it will taste bland. But too much salt tastes bad and makes us unhealthy.
It may be wise to use the black hat only once per thinking session, after coming up with ideas.

\subsection*{Optimism Hat Questions}
In general, the Optimism Hat is looking forward to the future. It can also be used to find the good effects of a past decision or event. It is crucial to wear the Optimism Hat for some time in either case.
When wearing the Optimism Hat, ask yourself the following questions.

\textbf{What are the benefits?} \\
When wearing the Optimism Hat, ask what the benefits are. Ask who the idea will benefit, and how the benefits come about. Be logical and give reasons. There may be savings in cost, improvements to function, or new opportunities. Find the value in an idea. If you cannot come up with benefits, the idea is not worth pursuing.

\textbf{Why should it work?} \\
Show clearly why an idea will work. Give reasons. If you cannot show the feasibility of an idea, it is likely not worth pursuing.

\subsection*{Balancing Caution and Optimism}
Either hat worn too often can cause problems. Excellent thinkers balance the time they spend wearing the Caution Hat and the Optimism Hat.

\subsection*{Caution and Optimism Hat Exercises}
\begin{enumerate}
	\item Someone suggests designing a car exclusively for women. Do some Caution Hat thinking, then some Optimism Hat thinking on this idea.
	\item Which of these arguments are valid Caution Hat arguments? Why? \\
		``Imposing fines for littering creates a police state'' \\
		``A written newsletter won't work because so many people can't read'' \\
		``People who tell lies get found out'' \\
		``Paying people more doesn't make them happier'' \\
		``If you don't study, you will fail the test''
	\item Put on the Optimism Hat and do some thinking about the Caution Hat.
\end{enumerate}

\section*{Idea Hat}

The Idea Hat is the most active hat. It is for creative thinking. There are five main uses of the green hat:

\textbf{1. Exploring} \\
Explore the situation and your own ideas, concepts, suggestions, and possibilities.

\textbf{2. Proposing and Suggesting} \\
Use the Idea Hat to make proposals or suggestions of any sort. Come up with possible decisions, suggestions for actions, or proposals to solve a problem. 

\textbf{3. Finding Alternatives} \\
Come up with alternative solutions. Just focus on coming up with them; use the Caution and Optimism Hats to analyze them.

\textbf{4. Forming New Ideas} \\
Sometimes, you need very new ideas. Use this hat to come up with new ideas, but do not worry about analyzing them.

\textbf{5. Provoking Ideas} \\
Use this hat to put forward unreasonable or silly ideas. Get your mind out of its usual track by coming up with absurd ideas. We will practice this in a later chapter.

\subsection*{Idea Hat Exercises}
\begin{enumerate}
	\item You own a pizza restaurant. Someone opens another pizza place next door. Put on your Idea Hat to come up with advertisement ideas.
	\item You need to draw a monster that lives in your closet. Put on your Idea Hat and decide how the monster can look.
\end{enumerate}

\section*{Meta Hat}
Use the Meta Hat to get an overview of your thinking process. The Meta Hat is often useful at the beginning of a thinking session. Ask yourself the following questions when wearing the Meta Hat.

\textbf{Where am I now?} \\
Ask where you are in your thinking. What is your focus? What are you trying to accomplish? Are you actively trying to do something?

\textbf{What is the next step?} \\
What should you do next in your thinking? You may suggest the use of a different hat.

\textbf{What is the plan?} \\
Instead of choosing just one next step, you can set out a whole program for thinking on the subject. You can decide what hats are best to wear at what stage, and how long you want to spend wearing each hat.

\textbf{What is the summary?} \\
Use the Meta Hat to gather all your notes and put together a summary of your thought process. It is at this point that you may wish to draw your conclusions.

\subsection*{Observation on the Meta Hat}
When wearing the Meta Hat, you are above the thinking, looking down at what is happening. Imagine watching your own thinking from a camera. A Meta Hat thinker may use 

\subsection*{Meta Hat Exercises}
\begin{enumerate}
	\item You are tasked with improving the road system in your town. Using your Meta Hat, plan how you might think through this problem.
	\item Timmy and his parents disagree about what time Timmy should get home in the evening. How should Timmy build his argument?
\end{enumerate}


\chapter{Thinking Hats in Sequence}

The thinking hats can be useful to switch gears while you think, and you can use them at sundry times to suggest a particular way of thinking or approaching a problem. But using the hats systematically can allow you to carefully construct an argument or idea.

There is no single correct order to use the hats in. The desired sequence will vary with your circumstances, but this chapter gives you some guidelines.

\textbf{1. Reusing a Hat} \\
Each hat can be used any number of times in a sequence.

\textbf{2. Optimism and Caution} \\
It is often better to use the Optimism Hat before the Caution Hat, since it is sometimes hard to be optimistic after being critical.

\textbf{3. Caution and Ideas} \\
The Caution Hat can be used to assess an idea. It can also be used to point out the weakness in an idea. This should be followed by the Idea Hat to try to overcome the idea's weakness.

\textbf{4. Final Assessment} \\
The Caution Hat should be used to make a final assessment of your conclusion. After you perform a logical assessment, don your Feeling Hat to notice how you feel about an idea.

\textbf{5. Start with Feelings} \\
If you have strong feelings about something, it can be helpful to start your thinking session wearing the Feeling Hat to get your feelings into the open.

\textbf{6. Start with Facts} \\
If you do not have strong feelings, it is good to start with the Fact Hat to collect and analyze information. 

\section*{Hat Sequence Exercises}

\begin{enumerate}
	\item Create a thinking hat sequence for each of the following situations: \\
	You have to plan a surprise party for a friend. \\~\\
	You want to teach someone how to bake a cake. \\~\\
	You are deciding what state your family should move to. \\~\\
	Someone asks what they can do to enjoy life more. \\~\\
	\item Think of a problem in your life. Devise and follow a thinking hat sequence to resolve that problem. Use all six thinking hats.
\end{enumerate}

\chapter{Five-Minute Format}

\ibox{\textbf{Conclusions} \\
	Thinking should always lead to a conclusion. This conclusion could be a better idea of what you're thinking about, an identification of needs, a specific answer to a question, or a solution to a problem.
}

A five-minute format can help you practice thinking. Setting a five-minute timer can be very important, as it forces you to move your thoughts forward. The following outline can help you think through a problem very quickly.

\textbf{First Minute} \\
During the first minute, identify the purpose of your thinking. Be clear about your focus and the outcome you need. Be clear about your situation. If you do not think you have enough information, set your own circumstances and outline your assumptions. For instance, you may say, ``I am assuming the person is 15 and this is the first time they have done this.''

\textbf{Next Two Minutes} \\
Don the Fact Hat. Explore the subject in terms of the information you have and the experiences you've had. Condense your ideas into a number of alternatives, and using the Idea Hat, come up with some courses of action or potential solutions. Ask yourself if there is an obvious answer. What are the common answers to your question? How can you put your wish into practical action? What other ways are there?

\textbf{Next Minute} \\
Spend this time making decisions. Ask what alternative is most likely to work. Which alternative best fits your needs and priorities?

\textbf{Final Minute} \\
Spend the last minute testing your decision. Go through reasons you think it will (or won't) work. Compare it with other solutions to show why your conclusion is better. Identify what you have learned while thinking about the subject.

\section*{Five-Minute Format Exercises}

Pick any 3 of these situations to do a five-minute format with.
\begin{enumerate}
	\item Your neighbors always park in front of your driveway, so you cannot use your garage. What can you do about it?
	\item A survey shows that people who eat far too much are unhealthy. What can be done?
	\item A girl feels that a teacher is unfair in his grading. What can she do?
	\item If you had to choose whether to live on land or on water, which would you choose?
	\item A friend wants to have a party, but his parents forbid it. What can you do?
	\item Your car breaks down in the middle of nowhere. How can you get home?
\end{enumerate}


\chapter{Series or Parallel}

There are two main ways of thinking. You can think one thing after another, walking along a path. Or you can pause and look around the garden. Imagine a one-lane road. One car goes after the other; the cars travel in series. But when a road has more lanes, cars can travel next to each other in parallel.

Parallel thinking gives us a valuable insight by answering this question: \textit{What else is there?} Series thinking gives us an insight to the question \textit{What is next?}

Both series thinking and parallel thinking are valuable. Sometimes, series thinking is called \textit{convergent thinking}, and parallel thinking is called \textit{divergent thinking}. Let's consider two math questions and the type of thinking that is best for them.

\textbf{What is the solution to 5+3?} \\
Using series thinking, we can conclude that by adding five to three, we get 8.

\textbf{What numbers add together to equal 8?} \\
This calls for parallel thinking. We remember from the previous problem that 5+3=8, but we also know that 4+4=8, 2+6=8, and 1+7=8.

\section*{Series or Parallel Exercises}

Use either series or parallel thinking on each of these problems. Write which you used and give an answer.

\begin{enumerate}
	\item Name as many uses as you can for a piece of paper. \\~\\~\\
	\item What process should I follow to boil an egg? \\~\\~\\
	\item What interesting places can I visit in your town? \\~\\~\\
	\item How can I drive to one of those places? \\~\\~\\
	\item What ideas do you have to help someone make friends at school? \\~\\~\\
	\item What is the best brand of ice cream? \\~\\~\\
	\item What are the benefits of riding a bike to work?
\end{enumerate}


\chapter{Logic and Perception}

























