
\chapter{Internet Rules}
\begin{enumerate}
	\item \textbf{Time Limit: } Multiply your age in years by five. This is the number of minutes you can use the internet for non-academic purposes each day.
	\item \textbf{Strangers: } Never use the internet to chat with strangers.
	\item \textbf{Personal Info: } Never share information about yourself or others.
	\item \textbf{Social Media: } Your parents must have a link to all your social profiles.
	\item \textbf{Photos: } Never share a photo of a person without permission.
	\item \textbf{Downloads: } A parent must approve every file download.
	\item \textbf{Location: } Devices may only be used and charged in public spaces.
	\item \textbf{Legality: } Never violate the law online (including age limits).
	\item \textbf{Bullying: } Never use the internet to harm another person. Report bullies.
	\item \textbf{Pornography: } If you see pornography in any form, immediately walk away from the device and tell your parents.
\end{enumerate}

These rules will keep you and others safe as you use the internet. Ask your parents if you need an exception to any of these rules.

\textbf{Discipline: } Internet access is a privilege, not a right. Your parents regularly check the internet history. Let y be your age in years. Let v be the number of times you've broken any internet rule. For each rule broken, you will lose internet access for (2$\times$v$\times$y). For example, a 13-year-old who breaks 2 rules at once will lose (2$\times$1$\times$13)+(2$\times$2$\times$13) = 78 days of internet access. If you bully another person repeatedly, you will lose social media access until you turn 18.

\textbf{Amnesty: } If you voluntarily tell your parents that you have broken rule 1, 3, 4, 5, 6, 7, or 9, your internet restriction will be halved. If you voluntarily tell your parents about breaking rule 2, 8, or 10, you will not lose internet access.


\chapter{The Need for Internet Safety}

\beforeyoubegin{List some ways you could stay safe in a crowd of people. Why are street smarts important?}

The internet is an extraordinarily powerful tool. But it can be a dangerous place. Imagine a busy outdoor market. If you know what you are doing and keep your wits about you, it's a fun, lively place. But if you don't know what you're doing, you could end up in trouble.

There is no need to be scared of a computer. But before you use the internet, you need to understand what dangers there are and how to keep yourself safe.

\section*{Rules and Laws}

Your parents set rules for your internet use for a number of reasons. They love you and want to keep you safe. Their rules are meant to protect you from harm as you use the internet.

There are also laws about the internet. Some of these laws say you can't use some websites until you are 13 or 18. Even though it is tempting to break these laws, obeying them will keep you safe.

Computers and internet services keep track of what you do on the internet. Doing wrong or illegal things on the internet means you is serious. You could be prohibited from using the internet again, or in some cases, you could go to jail.

Using the internet correctly will keep you and other people safe.

\section*{Dangers}

Most people on the internet are good people who mean no harm. But some people use the internet to hurt others. If you do not use the internet responsibly, people might steal your information, install harmful things on your computer, bully you, and send you inappropriate things.

You have a responsibility to protect yourself and your family on the internet.

\section*{Rule of Thumb}

If anything seems serious, suspicious, inappropriate, or harmful, \textbf{walk away from the device and tell your parents right away}. They will not be upset with you. They know how to handle bad things on the internet, and they will solve the problem. It is unwise to try and handle it yourself, and it could get you in trouble. Even expert internet users run into bad websites sometimes.

\section*{Exercises}

\begin{enumerate}
	\item 
\end{enumerate}


\chapter{Privacy}

The most important issue on the internet is privacy. 

\textbf{When in doubt, do not share any information on the internet}.

If something looks serious, always ask a responsible adult before you share any information or click anything. Some information is okay to share, but most is not. Always ask your parents before sharing anything labeled ``Do Not Share''.

\begin{multicols}{2}
	\tbox{Okay to Share}{
		A random username \\
		The country you live in \\
		The state you live in \\
		Your favorite song \\
		Artwork you made \\
		Your opinion
	}
	\ibox{If you're ever not sure, ask your parents if you can share something.}
\newcolumn
	\tbox{Do Not Share}{
		Info that isn't yours \\
		Photos of faces \\
		Any password \\
		Your name \\
		Your address \\
		Your age \\
		Your birthday \\
		Social security numbers \\
		Credit card numbers \\
		A phone number \\
		The city you live in \\
		The school you go to \\
		Vacation photos or information
	}
\end{multicols}

Occasionally, someone may want to share a photo of you online. Your parents and other responsible adults should always ask if you are comfortable with your photo being shared. You are always allowed to say no. Teachers or other adults at school may take your photo and use it only for school-related things. Otherwise, never let someone take your photo at school.
If someone who is not a trusted adult wants to take or share your photo, always refuse. Never allow a stranger to take your photo without first asking your parents.


\chapter{Communication}

The internet is a great way to stay in touch with people around the world. Until you are 13, your parents will help you chat with friends and family members using a family account. Once you are 13, you may join social media websites. 

\section*{Email}

While you are under 13 years old, you will have a monitored email address. Your parents should always have access to this account. Once you are 13, you may wish to create a personal email address. If your parents have any reason to suspect you are using your email address inappropriately, they will read your emails or delete your email address.

You will get emails from people you do not know. If you don't recognize the sender, never open the email. If an email ever asks for your personal information or money, talk to your parents about it immediately. 

\section*{Bullying}

You should communicate online in the same way you would communicate in-person. If you wouldn't say something to someone's face, never say it to them online. Do not ever use social media to hurt other people.

\textbf{If you use the internet to hurt others, you will not be allowed on social media until you turn 18}. Bullying over the internet is against the law, and it will not be tolerated in this family in the slightest degree.

Never allow yourself to be bullied over the internet. If someone is using social media to hurt you, tell your parents immediately. Reporting bullying is good for the bully and the victim, and it keeps the internet safer.


\chapter{Time Limits}

