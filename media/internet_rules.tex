\begin{enumerate}
	\item \textbf{Time Limit: } Multiply your age in years by five. This is the number of minutes you can use the internet for non-academic purposes each day.
	\item \textbf{Strangers: } Never use the internet to chat with strangers.
	\item \textbf{Personal Info: } Never share information about yourself or others.
	\item \textbf{Social Media: } Your parents must have a link to all your social profiles.
	\item \textbf{Photos: } Never share a photo of a person without permission.
	\item \textbf{Downloads: } A parent must approve every file download.
	\item \textbf{Location: } Devices may only be used and charged in public spaces.
	\item \textbf{Legality: } Never violate the law online (including age limits).
	\item \textbf{Bullying: } Never use the internet to harm another person. Report bullies.
	\item \textbf{Pornography: } If you see pornography in any form, immediately walk away from the device and tell your parents.
\end{enumerate}

These rules will keep you and others safe as you use the internet. Ask your parents if you need an exception to any of these rules.

\textbf{Discipline: } Internet access is a privilege, not a right. Your parents regularly check the internet history. Let y be your age in years. Let v be the number of times you've broken any internet rule. For each rule broken, you will lose internet access for (2$\times$v$\times$y). For example, a 13-year-old who breaks 2 rules at once will lose (2$\times$1$\times$13)+(2$\times$2$\times$13) = 78 days of internet access. If you bully another person repeatedly, you will lose social media access until you turn 18.

\textbf{Amnesty: } If you voluntarily tell your parents that you have broken rule 1, 3, 4, 5, 6, 7, or 9, your internet restriction will be halved. If you voluntarily tell your parents about breaking rule 2, 8, or 10, you will not lose internet access.